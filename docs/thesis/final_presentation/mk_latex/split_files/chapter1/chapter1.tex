\chapter{はじめに}\label{ux306fux3058ux3081ux306b}

    \section{研究の目的}\label{ux7814ux7a76ux306eux76eeux7684}

    西谷研究室ではプログラミング言語Rubyを使用して,言語学習や卒業研究を行なっている. つまり, 3年生はRubyの習得が早ければ自分の研究の為の時間を確保できる. しかし,個人で言語学習を行う際のハードルは高いものである[1]. その原因として,言語に対しての知識がない初心者にとって環境構築や学習の教材選びは非常に困難であることが要因の一つではないかと考えられる.環境構築なしで教材選びも必要ない言語学習サービスには Progate[2] や codecademy[3] 等があるが,環境構築という部分を排斥しているが故にそのサービス外での学習に一定の壁が生じているのではないかと考える.
また,近年ではコードをGitで管理することで,チーム内でのコード編集の履歴をメンバーがそれぞれ確認することが出来るので,チームでの開発が管理しやすくなった[4]. このような開発では他人の書いたコードを読み解き,それに適切な形で自分のコードを書き加える能力が必要となる. そういった意味でも,近年の言語学習は動くだけのコードを書くのではなく言語が持つスタイルについてもきちんと学ぶ必要があると考える.
そこで本研究では,環境構築の自動化に加え,Rubyの体系的な言語学習とプログラミングスタイルの学習ができるアプリケーションを開発し,Rubyを容易に学習できる環境を整え,実践的な学習を提供することで,個人のスキル向上の効率化を目指す.

    \section{研究の動機}\label{ux7814ux7a76ux306eux52d5ux6a5f}

    本研究室に在籍していた和田によって開発されたタイピングアプリケーションeditor\_learnerの再開発を検討していたが,本研究室で行う開発に必要なスキルの習得に焦点を当てた開発に切り替えた.ここでのスキルとは,動くだけでなくコーディング規約に則ったコードが書けることや開発に必要な基本的なツールの使い方の習得のことである.

    
