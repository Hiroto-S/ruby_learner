\chapter{ruby\_learnerの概要}
\section{Install/Uninstall}\label{install/uninstall}
gemによるインストール方法は以下の通りである.

\section{動作環境}\label{operating_environment}
Rubyのversionが2.4.0以上でなければ動かない.理由としては,gemに格納されているパスを正しいく受け渡しできないからである.2.4.0以下で動作させるためにはeditor\_learnerの最新versionのみを入れることによって動作することが確認できている.さらに,open\_terminalがMacOSXでのみ動作するのでMacOSXを使用すること.

\section{sequential\_check}\label{sequential_check}
特別な初期設定はほとんどないが起動方法は以下の通りである,

\subsection{教材}\label{text}
errorが出た場合は以下の方法を試してください

\subsection{課題}\label{question}
errorが出た場合は以下の方法を試してください

\subsection{評価}\label{evaluation}
errorが出た場合は以下の方法を試してください

\subsection{normal\_mode}\label{nomal_mode}
errorが出た場合は以下の方法を試してください

\subsection{manual\_mode}\label{manual_mode}
errorが出た場合は以下の方法を試してください

\subsection{last}\label{last}
errorが出た場合は以下の方法を試してください

\subsection{next}\label{next}
errorが出た場合は以下の方法を試してください

\subsection{workshop}\label{workshop}
errorが出た場合は以下の方法を試してください

\section{restore}\label{restore}
特別な初期設定はほとんどないが起動方法は以下の通りである,

\subsection{check}\label{check}
errorが出た場合は以下の方法を試してください

\subsection{open}\label{open}
errorが出た場合は以下の方法を試してください

\subsection{refresh}\label{refresh}
errorが出た場合は以下の方法を試してください

\section{pair\_popup}\label{pair_popup}
特別な初期設定はほとんどないが起動方法は以下の通りである,

\section{install\_emacs}\label{install_emacs}
特別な初期設定はほとんどないが起動方法は以下の通りである,

\section{emacs\_key}\label{emacs_key}
特別な初期設定はほとんどないが起動方法は以下の通りである,

\subsection{string}\label{string}
errorが出た場合は以下の方法を試してください

\subsection{image}\label{image}
errorが出た場合は以下の方法を試してください

\section{theme}\label{theme}
特別な初期設定はほとんどないが起動方法は以下の通りである,

\section{copspec}\label{copspec}
特別な初期設定はほとんどないが起動方法は以下の通りである,
