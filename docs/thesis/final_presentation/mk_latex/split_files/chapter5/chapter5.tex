\chapter{使用者からの評価}\label{evaluation}
実際に9名のRuby未学習者に本研究で作成したアプリケーションを使用してもらい,本研究の評価を行う.

\section{肯定的な評価}\label{list}
以下は被験者の評価の内,肯定的な評価をリスト状にしたものである.
\begin{enumerate}
\def\labelenumi{\arabic{enumi}.}
\tightlist
\item
  Rubocopで優れた書き方へ誘導してくれる点
\item
  一つの画面でテキストと課題に取り組める点
\item
  CUIでの操作に慣れることが出来る点
\item
  Emacsのキーバインドを習得できる点
\item
  課題の答えをいつでも確認できる点
\end{enumerate}

\section{否定的な評価}\label{list}
以下は被験者の評価の内,否定的な評価をリスト状にしたものである.
\begin{enumerate}
\def\labelenumi{\arabic{enumi}.}
\tightlist
\item
  課題に関数の使い方などのヒントが欲しい
\item
  他言語バージョンが欲しい
\item
  Rubocopのコーディング規約を変更したい
\end{enumerate}

\section{考察}\label{list}
本アプリケーションはCUI操作に慣れることが出来る点や使用者のコードの評価方法について肯定的な評価を得た.操作に慣れることでコマンド操作のみで学習を行えることは学習効率の向上に繋がると考えられる.使用者のコードの評価にはRSpecとRubocopの2つを用いたが,その組み合わせで使用者のコードは当事者目線ではなく定量的な評価を受けることが可能であったため,このような評価に繋がったと考える.以上から,本アプリケーションを用いる事で,使用者は効率的な学習が出来ていると考えられる.
しかし,今回の実験以外で本アプリケーションを使用したいかという問いに対して,使用したいと答えた人は9人中2人のみであった.
本アプリケーションには多くの改善点がある.そのうちの一つとして,モチベーション維持を目的とする機能が必要であると考える.その機能に関しては本研究室の高橋が開発中のruby\_learner\_checkerで実装・検証される.またコーディング規約や課題の難易度の部分で,個人個人が求めている学習の詳細なスタイルにバラつきが見受けられるため,個人でのカスタマイズ機能の追加も検討する.
